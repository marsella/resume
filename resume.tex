\documentclass{article}
\usepackage[margin=1in]{geometry}
\usepackage{enumitem}

\setlength\parindent{0pt}

\newcommand{\textbox}[1]{\parbox{.333\textwidth}{#1}}

\begin{document}
\pagestyle{empty}
\noindent\textbox{\textbf{Marcella Hastings}\hfill}\textbox{\centering \texttt{mhast@seas.upenn.edu}}\textbox{\hfill marcellahastings.com}

\hrulefill

\textsc{Education}
\begin{itemize}[label={}]
  \item \textbf{University of Pennsylvania}, Philadelphia, Pennsylvania USA \\
    M.S., Ph.D., Department of Computer Science \textit{(expected December 2020)}\\
    Advisor: Nadia Heninger. GPA: 3.90
    
  \item \textbf{Tufts University}, Medford, Massachusetts USA \\
    B.S., Computer Science and Mathematics, May 2015 \\
    \textit{Summa Cum Laude}
\end{itemize}

\textsc{Publications}

\textit{Refereed Conference Proceedings}
\begin{itemize}[label={}]
\item SoK: General Purpose Compilers for Secure Multi-Party Computation. 
Marcella Hastings, Brett Hemenway, Daniel Noble, and Steve Zdancewic.
In \textit{40th IEEE Symposium on Security and Privacy} (Oakland `19). May 2019.
\item The Proof is in the Pudding: Proofs of Work for Solving Discrete Logarithms.
Marcella Hastings, Nadia Heninger, Eric Wustrow.
In \emph{Financial Cryptography and Data Security} (FC `19). February 2019.
\item Measuring Small Subgroup Attacks on Diffie-Hellman. 
Luke Valenta, David Adrian, Antonio Sanso, Shaanan Cohney, Joshua Fried, Marcella Hastings, J. Alex Halderman, Nadia Heninger. 
In \textit{Network and Distributed System Security Symposium} (NDSS `17). February 2017.
\item Weak Keys Remain Widespread in Network Devices. 
Marcella Hastings, Joshua Fried, and Nadia Heninger. 
In \textit{Proceedings of the 2016 ACM on Internet Measurement Conference} (IMC `16). November 2016.
\end{itemize}


\textsc{Work Experience}
\begin{itemize}[label={}]
  \item \textbf{Software Applications and Innovations Lab}, Boston, MA USA, May 2019 - Present\\
  \textit{Research Intern.} Implementing feature libraries and a cryptographically secure protocol for generating preprocessing data in the JIFF framework for secure multi-party computation.
  \item \textbf{MIT Lincoln Laboratory}, Lexington, MA USA, May - August 2014 \\
  \textit{Research Intern.} Developed an end-to-end prototype for a cryptographically secure mechanism for authentication from a single fortified device.
  \item \textbf{Google}, New York, NY USA, June - August 2013\\
  \textit{Engineering Practicum Intern.} Designed and implemented a client-facing interface and implementation for a saving filters feature with the DoubleClick for Publishers team.
\end{itemize}

\textsc{Invited Talks}
\begin{itemize}[label={}]
\item \emph{General purpose compilers for secure multi-party computation}\\
  DC Area Crypto Day, December 2018 \\
  Theory and Practice of Multi-Party Computation Workshops, June 2019
\end{itemize}

\textsc{Teaching}
\begin{itemize}[label={}]
\item Designed course materials, taught hands-on programming labs, developed grading software, managed teaching assistants, and held regular office hours for undergraduate and graduate-level courses.

\item \textbf{University of Pennsylvania:}
CIS 331: Introduction to Networks and System Security, Spring 2017.
CIS 556: Cryptography, Fall 2016.
GEMS Computer Science Workshop, Summer 2017.
\item \textbf{Tufts University:} 
COMP 170: Theory of Computation, Spring 2015.
COMP 50: Problem-Solving by Computer, Fall 2013.
COMP 11: Introduction to Computer Science, Fall 2012 - Spring 2015.
\end{itemize}

\end{document}

